\documentclass{tufte-handout}

%\geometry{showframe}% for debugging purposes -- displays the margins

\usepackage{amsmath, amsfonts, amsthm, amssymb}

\newtheorem{theorem}{Theorem}[section]
\newtheorem{definition}{Definition}[section]
\newtheorem{lemma}{Lemma}[section]
\newtheorem{proposition}{Proposition}[section]
\newtheorem{corollary}{Corollary}[section]
\newtheorem{example}{Example}[section]

% Set up the images/graphics package
\usepackage{graphicx}
\setkeys{Gin}{width=\linewidth,totalheight=\textheight,keepaspectratio}
\graphicspath{{graphics/}}

\title{Real Analysis II}
\author[Feiyang Wu]{Feiyang Wu}
% \date{}  % if the \date{} command is left out, the current date will be used

% The following package makes prettier tables.  We're all about the bling!
\usepackage{booktabs}

% The units package provides nice, non-stacked fractions and better spacing
% for units.
\usepackage{units}

% The fancyvrb package lets us customize the formatting of verbatim
% environments.  We use a slightly smaller font.
\usepackage{fancyvrb}
\fvset{fontsize=\normalsize}

% Small sections of multiple columns
\usepackage{multicol}

% Provides paragraphs of dummy text
\usepackage{lipsum}

% These commands are used to pretty-print LaTeX commands
\newcommand{\doccmd}[1]{\texttt{\textbackslash#1}}% command name -- adds backslash automatically
\newcommand{\docopt}[1]{\ensuremath{\langle}\textrm{\textit{#1}}\ensuremath{\rangle}}% optional command argument
\newcommand{\docarg}[1]{\textrm{\textit{#1}}}% (required) command argument
\newenvironment{docspec}{\begin{quote}\noindent}{\end{quote}}% command specification environment
\newcommand{\docenv}[1]{\textsf{#1}}% environment name
\newcommand{\docpkg}[1]{\texttt{#1}}% package name
\newcommand{\doccls}[1]{\texttt{#1}}% document class name
\newcommand{\docclsopt}[1]{\texttt{#1}}% document class option name

\begin{document}

\setcounter{section}{1} 

\maketitle% this prints the handout title, author, and date

\begin{abstract}
\noindent This documents reviews important topics in Real Analysis.
\end{abstract}

\section{Basics of Abstract Measure}
\begin{definition}[$\sigma$-algebra]
    For a given set $X$, a collection $\mathcal{M}$ of subsets of $X$ is called the $\sigma$-algebra if 
    \begin{enumerate}
        \item $\emptyset\in \mathcal{M}$,
        \item $A\in \mathcal{M} \Rightarrow A^c \in \mathcal{M}$,
        \item If $A_1, A_2, \dots$ is a sequence in $\mathcal{M}$, then $\bigcup^\infty_{j=1}A_j \in \mathcal{M}$, and consequently, $\bigcap^\infty_{j=1}A_j \in \mathcal{M}$.
    \end{enumerate}
\end{definition}

\begin{definition}[Measure]
    A measure on $\mathcal{M}$ is a function $\mu:\mathcal{M} \rightarrow [0,+\infty]$ such that $\mu(\emptyset)=0$, and if $A_1, A_2, \dots$ is a sequence in $\mathcal{M}$ and pairwise disjoint, then 
    \[
    \mu\left(\bigcup^\infty_{j=1}A_j \in \mathcal{M}\right) = \sum^{\infty}_{j=1} \mu(A_j).
    \]
\end{definition}
\subsection{Basic Consequences}
\begin{enumerate}
    \item If $A,B\in \mathcal{M}$, and $A\subseteq B$, then $\mu(A) \leq \mu(B)$. (due to $B=A\cup B\backslash A$).
    \item If $A_1, A_2, \dots \in \mathcal{M}$, then 
    \[
    \mu\left(\bigcup^\infty_{j=1}A_j \in \mathcal{M}\right)
    \leq 
    \sum^{\infty}_{j=1} \mu(A_j).
    \]
    \begin{proof}
        Put $B_1=A_1, B_2=B_1\backslash A_1, B_3\backslash A_2\cup A_1, \dots, B_n=A_n\backslash \bigcup^{n-1}_{j=1}A_j,\dots$ and we have $B_j\in \mathcal{M}$, and $\bigcup^{\infty}_{j=1} B_j = \bigcup^{\infty}_{j=1} A_j$, and $B_j$s are pairwise disjoint. Hence,
        \[
        \mu \left( \bigcup^{infty}_{j=1} A_j \right)
        =\mu \left( \bigcup^{infty}_{j=1} B_j \right) \leq 
        \sum^{\infty}_{j=1} \mu(B_j) \leq \sum^{\infty}_{j=1} \mu(A_j).
        \]
    \end{proof}
\end{enumerate}

\subsection{Continuity properties}
\begin{proposition}[Continuity of Measure]
    \mbox{}
    \begin{enumerate}
        \item If $A_1, A_2, \dots \in \mathcal{M}$, $A_j \subseteq A_{j+1}$, if $A = \bigcup^{\infty}_{j=1} A_j$, then 
        \[
        \mu(A) = \lim_{j\rightarrow \infty} \mu(A_j).
        \]
        \item  Suppose $\mu(A_1) < +\infty$, and if $A_1, A_2, \dots \in \mathcal{M}$, $A_j \supseteq A_{j+1}$, then 
        \[
        \mu(\bigcap^{\infty}_{j=1} A_j) = \lim_{j\rightarrow \infty} \mu(A_j).
        \]
    \end{enumerate}
\end{proposition}
\begin{proof}
    For (1): Put $B_1=A_1, B_2=B_1\backslash A_1, B_3\backslash A_2\cup A_1, \dots, B_n=A_n\backslash \bigcup^{n-1}_{j=1}A_j,\dots$ and we have $B_j\in \mathcal{M}$, and $\bigcup^{\infty}_{j=1} B_j = \bigcup^{\infty}_{j=1} A_j$, and $B_j$s are pairwise disjoint. Hence,
    \[
    \mu(A) = \sum^{\infty}_{j=1} \mu(B_j) = \lim_{k\rightarrow \infty} \sum_{j=1}^{k}\mu(B_j) = \lim_{k\rightarrow \infty} \mu\left( \bigcup^{k}_{j=1} B_j\right) = \lim_{k\rightarrow \infty} \mu(A_k).
    \]
    For (2): Consider $B_j = A_1\backslash A_j$. Then we have $B_1 \subseteq B_2 \subseteq \dots$. We then apply (1), for which we get
    \[
    \mu\left( \sum_{j=1}^\infty B_j \right) = \lim_{j\rightarrow }\mu(B_j) = \lim_{j\rightarrow \infty} \mu(A_1\backslash A_j).
    \]
    Note that $\mu\left( \sum_{j=1}^\infty B_j \right) =\mu(A_1 \backslash \bigcap_{j=1}^\infty A_j) = \mu(A_1) - \mu(\bigcap_{j=1}^\infty A_j)$. On the other hand, $ \lim_{j\rightarrow \infty} \mu(A_1\backslash A_j)=\lim_{j\rightarrow \infty}(\mu(A_1) - \mu(A_j))$. Notice that we can only say 
    $\mu(A\backslash B) = \mu(A) - \mu(B)$ if $\mu(B)<+\infty$. So that we have
    \[
    \lim_{j\rightarrow \infty} \mu(A_j) = \mu\left(\bigcap^\infty_{j=1} A_j\right) \text{ if } \mu(A_1) < +\infty.
    \]
\end{proof}
\begin{example}[Failure if $\mu(A_1) = \infty$]
    If $A_k=[k,\infty]$, then $\mu(A_1)=\infty$, but $\bigcap^\infty_{j=1} A_j = \emptyset$, and the Lebesgue measure $m(A_k) = \infty$ for every $k$.
\end{example}

\section{Operator Theory}
\subsection{Dual space}


\bibliography{sample-handout}
\bibliographystyle{plainnat}



\end{document}
